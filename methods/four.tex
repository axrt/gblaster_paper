\subsection{Regression model quality}
\label{regmodqual}
It is clear that the reliability of the logistic regression models obtained as
described in \ref{simsearch}, should be controlled for. Consider a ratio
$\dfrac{n_{g_{uncertain}}}{n_{g_{core}}+n_{g_{outsider}}}$ where
$n_{g_{uncertain}}$ is the total number of $g$s predicted as ``uncertain'' by a
regression model, and similarly, $n_{g_{core}}$ and $n_{g_{outsider}}$ are the
total number of $g$s classified as ``core'' and ``outsider'' respectively.
Intuitively, the closer a genome $A$ is to genome $B$ on the tree of life (say,
genera from the same order by conventional taxonomic classification), the more
difficult it is to confidently assign individual $g$s from a mix $\bmzpab$ back
to the corresponding clusters $A$ and $B$ due to a relatively higher genome's
$G_A$ and $G_B$ genome sequence constitution. Thus, the subjective "difficulty"
of a reliable assignment may vary grately between certain pairs of related
genomes allowing for reliable assignment only in some cases.

In this study we propose the common McFadden's $R^2$ \cite{McFadden1974} as a
measure of a logistic regression model quality. We chose to only consider
models with $R_{McFadden}^2$ values larger than 0.73 - an arbitrary selected
cutoff, which roughly corresponds to the starting point of the linear growth on
the plot of $R_{McFadden}^2$ values sorted ascending for all models obtained
previously (see Supplement). We would like to emphasize that the choice of the
cutoff should depend on the nature of the study and be derived from the model
quality distribution.
