\subsection{Domain analysis}
The term $g_{outsider}$ introduced in \ref{regmodqual} may be extended to
cover multiple comparisons: a $g$ may be classified as an ``outsider'' by
multiple models simultaniously. We propose a term $g_{outsider}^{multiple}$ to
denote such $g$, and $g_{outsider}^{once}$ denote a special case when a $g$ was
classified as ``outsider'' only by one model. We aregue that
$g_{outsider}^{multiple}$ are stronger candidates for an HGT event, or an event
of a detected contamination of a number of genome assembiles. To gain a better
understanding of the nature of the $g_{outsider}^{multiple}$s we chose to
analyse their putative domain structure.

We used PFAM \cite{Finn2014} to carry out the domain analysis. For each
genome, we compared the number of genes with each discovered PFAM domain
between the core and outsider multiple sequences. We used a chi-square test to
assess whether any domains were significantly over- or underrepresented among
outsider multiple sequences.  We considered p-values of 5.5 x 10-7 to be
statistically significant (a Bonferroni correction for multiple comparisons was
applied to account for 89688 domain/species combinations used in the analysis).
Note that technically this is not a strict statistical procedure, since similar
genes will have similar arrays of bit scores and our classification of these
genes will not be independent. Similarly, this does not distinguish multiple
HGT events of sequences containing the same PFAM domain from a single HGT event
followed by multiple gene duplications. Main findings were evaluated manually.
