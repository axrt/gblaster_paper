\documentclass{article}
\usepackage{amsmath}    
\usepackage{leftidx}
\begin{document}
Let $G_A, G_B$ denote genomes $A$ and $B$ respectively,
and let a ``genome'' be a finite set of all possible PPSs
(the length of a PPS in this study is subject to constraints
$150 \leq length(PPS) \leq 1e4$) that can be derived from it as in:

\begin{equation}
G_A = \{g_A^1, g_A^2, \ldots, g_A^{L_A}\} = \{g_A^i\}_{i=1}^{L_A}
\end{equation}
where $g_A^i$ is a PPS from genome $A$ with index $i$, $L_A$ is the length of
the genome $A$.
Should we choose to search the most similar sequence for some $g_A^i$ in some
$G_B$ using a BLASTP algorithm implementation, we will end up with a (possibly
empty) set of best hits. Here we are only interested in the topmost hit's
bit-score denoted $b_{A \rightarrow B}^i$. More formally:

\begin{equation}
B(g_A^x, G_B) = BLASTP(g_A^x, G_B)=\left \{
\begin{aligned}
&b_{A \rightarrow B}^i\\
&0, && \text{if no hits found}
\end{aligned} \right.
\end{equation} 

Suppose we are provided with a finite set of genomes \\
$S = \{G_1, G_2, \ldots, G_M\} = \{G_j\}_1^M$, where $M$ is the number of
genomes in the set (64 total in this study). Should we choose to search for
sequence similarity for some $g_A^x$ in each genome, we will be provided with a
vector of best hit bit-scores:

\begin{equation}
V_{g_A^x} = (B(g_A^x, G_1), B(g_A^x, G_2), \ldots, B(g_A^x, G_M)) = (B(g_A^x,
G_j)_{j=1}^M)
\end{equation}

Obtained for each $g$ in every $G$ and ``stacked'' together, the $V$ vectors
form a matrix $BM$ of size $N*M$, where $N=\sum_{i=1}^k L_{G_i}$ and $M$,
again, - the total number of genomes in the study. For further analysis we chose
to ``filter'' $BM$ to only include $V$s which had at least 3 distinct hits
(``distinct'' here means the bit-score of a similarity search $B(g_A^x,
G_B)>0$), and denote such ``filtered'' matrix: $BM^+$.

As the last step of data preparation we suggested that the bit-score values be
scaled as follows. $b_{A \rightarrow A}^x = B(g_A^x, G_A)$ is essentially a
result of a search for the best hit for $g_A^x$ within genome $A$, which is
$g_A^x$ itself, and the bit-score of such a match is the maximum possible for
the given $g_A^x$. We will thus perform the scaling as for each $b$ in $BM^+$:

\begin{equation}
\leftidx{^0}b_{A \rightarrow B}^x = \dfrac{b_{A \rightarrow B}^x}{b_{A \rightarrow
A}^x} 
\end{equation}

and the resulting scaled materix we denote $\leftidx{^0}BM^+$. This matrix and
its slices, corresponding to $g$s from individual subsets of genomes, we will
be using for further analysis (see Attachment).

\end{document}
