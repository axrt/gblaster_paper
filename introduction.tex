\section{Introduction}
\label{intro}
Horizontal gene transfer (HGT) is widely accepted as an important factor in the
evolution of prokaryotic organisms \cite{Ochman2000}. At the
domain-of-life-scale, a number of HGT events between prokaryotes and
eukaryotes, and even between different eukaryotic species have been described
[needs link], and in some cases have revealed molecular mechanisms that
facilitate transmission \cite{Soucy2015}. For example, agrobacteria are widely
used in modern biotechnology because of their natural ability to transfer some
of their genes horizontally into plant genomes \cite{Chilton1977}. Plants with
genes of bacterial origin are thus not uncommon \cite{Kyndt2015, Matveeva2012,
Matveeva2014}. Another peculiar case is the acquisition of genes by arthropods
from bacterial endosymbionts. For instance, the genome of the
\textit{Callosobruchus chinensis} beetle contains around 30\% of the
\textit{Wolbachia} genome \cite{Nikoh2008}. Viruses may also contribute to
lateral gene transfer \cite{Drezen2017}. A particular gene of retroviral origin
encoding a protein called ``syncytin'' plays an important role in mammalian
placenta morphogenesis \cite{Mi2000}. There is some evidence that HGT can occur
between different eukaryotic species \cite{Soucy2015}, for example between
certain parasitic plants and their hosts
\cite{Yoshida2010,Xi2012,Zhang2013,Zhang2014}, but overall our knowledge of
this field is scarce. There is ongoing controversy about the extent to which
HGT plays a role in eukaryotic evolution. Numerous examples of HGT were found
in different eukaryotic species, including humans by Crisp et al
\cite{Crisp2015}. However, other researchers failed to confirm some of the
published HGT findings \cite{Salzberg2017}. A study by Ku et al concluded that
even the gene transfer from bacteria to eukaryotes happens only occasinally and
coincides with major evolutionary transitions such as the origin of
chloroplasts and mitochondria \cite{Ku2015}. It should be noted that Ku et al
analyzed eukaryotic gene families with prokaryotic homologs and used clustering
and phylogenetic analysis. Crisp et al used differences in blast bit-scores
between prokaryotic and non-prokaryotic best hits of eukaryotic genes as a
criterion for the identification of the putative HGT events. Thus, these
studies are quite different by design, including the formulated requirements
for HGT. Remarkably, the scientific community is yet to decide what may
constitute sufficient evidence to consider a gene to have been obtained through
an HGT event. Here we present a novel regression-based approach for the
identification of candidate HGT events in genomic sequences, and report our
findings of putative HGT events in 61 genomes covering different branches of
the tree of life.
