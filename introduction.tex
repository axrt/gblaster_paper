\section{Introduction} \label{intro} Horizontal gene transfer (HGT) is widely
accepted as an important factor in the evolution of prokaryotic organisms
\cite{Ochman2000}. At the domain-of-life-scale, a number of HGT events between
prokaryotes and eukaryotes, and even between different eukaryotic species have
been described \cite{Danchin2016a}, and in some cases have revealed molecular
mechanisms that facilitate gene transmission \cite{Soucy2015}. For example,
agrobacteria are widely used in modern biotechnology because of their natural
ability to transfer some of their genes horizontally into plant genomes
\cite{Chilton1977}. Plants with genes of bacterial origin are thus not uncommon
\cite{Kyndt2015} \cite{Matveeva2012} \cite{Matveeva2014}. Another peculiar case
is the acquisition of genes by arthropods from bacterial endosymbionts. For
instance, the genome of the \textit{Callosobruchus chinensis} beetle contains
around 30\% of its symbiont's \textit{Wolbachia} genome \cite{Nikoh2008}.
Viruses may also contribute to lateral gene transfer \cite{Drezen2017}. A
particular gene of retroviral origin encoding a protein called ``syncytin''
plays an important role in humans, apes and Old Worls monkeys placenta
morphogenesis \cite{Mi2000}. Arc, a gene that is also of retroviral origin,
plays in important role in the nervous system of some animals
\cite{Pastuzyn2018}. There is some evidence that HGT can occur between
different eukaryotic species \cite{Soucy2015}, for example between certain
parasitic plants and their hosts \cite{Yoshida2010} \cite{Xi2012}
\cite{Zhang2013} \cite{Zhang2014}, but overall our knowledge of this field is
scarce. There is ongoing controversy about the extent to which HGT plays a role
in eukaryotic evolution. Numerous examples of HGT were found in different
eukaryotic species, including humans by Crisp et al \cite{Crisp2015}. However,
other researchers failed to confirm some of the previously published HGT
findings \cite{Salzberg2017}. Thus, a study by Ku et al concluded that even the
gene transfer from bacteria to eukaryotes happens only occasionally and
coincides with major evolutionary transitions such as the origin of
chloroplasts and mitochondria \cite{Ku2015}. It should be noted that Ku et al
analyzed eukaryotic gene families with prokaryotic homologs and used clustering
and phylogenetic analysis. Crisp et al used differences in BLAST bit-scores
between prokaryotic and non-prokaryotic best hits of eukaryotic genes as a
criterion for the identification of the putative HGT events. The abovementioned
studies differ in their approach and experimental design.  Moreover, they do
not adhere to a common set of features reqiured to spot an HGT event. This, in
our opinion, highlights a problem: the scientific community is yet to decide
what evidence should be sufficient to consider a gene to have been obtained
through an HGT event. Here, we present a novel regression-based approach for
the identification of candidate HGT events in genomic sequences.  Using this
approach, we identified putative HGT events in 61 genomes covering different
branches of the tree of life.
