\section{Abstract}
\label{abstract}
\subsection{Background}
Genes within a single genome may have different phylogenetic histories: most
genes were passed down ``vertically'' from the direct ancestors of the
organism, while others may have been acquired via horizontal gene transfer
(HGT) from other species. HGT is common between prokaryotes, but has been
observed between prokaryotes and eukaryotes, and even between different
eukaryotic species. HGT events may lead to substantial HGT evolutionary
consequences. \todo{don't like this "evolutionary consequences" thing}
Nevertheless, no consensus\todo{don't like this consensus situation. consensus
where exactly? in this study? among people who chose to dedicate their lives to
this dumb stuff?} as been reached neither on the necessary criteria
required to confirm HGT, nor about the scale on which HGT may occur outside the
prokaryotic domains of life.
\subsection{Results}
We propose a novel regression analysis-based approach to identify putative HGT
events in genomic sequences. Using this technique we confirmed \todo{confirmed
is a bad word. how the fk did we do that? detected?} some of the previously
reported cases of HGT (such as the transfer of ankyrin-encoding sequences from
eukaryotes to their symbiotic bacteria or\todo{why 'or'? and?} the acquisition
of thaumatin by Caenorhabditis) and predicted a number of novel candidates for
HGT (such as the putative acquisition of MaoC dehydratase domain containing
proteins by Lokiarchaeota).
\subsection{Conclusions}
Our approach may be useful to discover candidate sequences with
``non-vertical'' phylogenetic histories, implicated in HGT.
\todo{I am not sure is we want to have this ``limitation'' thing here..}
Limited by the computational complexity of the algorithm described here, we
were able to compare only a moderate number of smaller genomes. Larger studies
however will improve the recall and coverage leading to further HGT event
discoveries.
