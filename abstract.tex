\section{Abstract}
\label{abstract}

\subsection{Background}
\label{bg}
Genes within a single genome may have different phylogenetic histories: most
genes were passed down ``vertically'' from the direct ancestors of the
organism, while others may have been acquired via horizontal gene transfer
(HGT) from other species. HGT is common between prokaryotes, but has been
observed between prokaryotes and eukaryotes, and even between different
eukaryotic species. HGT events may lead to substantial deviations to otherwise
expected evolution.
Nevertheless, no consensus among the scientific community has been reached
neither on the criteria of HGT detection, nor about the scale on which HGT may
occur outside the prokaryotic domains of life.

\subsection{Results}
\label{res}
We propose a novel regression-based approach to identify putative HGT
events in genomic sequences. Using this technique we were able to detect some
of the previously reported HGT candidates (such as the transfer of
ankyrin-encoding sequences from eukaryotes to their symbiotic bacteria
and the acquisition of thaumatin by \textit{Caenorhabditis}) and predicted a
number of novel candidates for HGT (such as the putative acquisition of MaoC
dehydratase domain containing protein).

\subsection{Conclusions}
\label{concl}
Our approach may be useful to discover candidate sequences with
``non-vertical'' phylogenetic histories, implicated in HGT. We also propose a
set of genes likely taking part in HGT in accordance with the proposed model.
