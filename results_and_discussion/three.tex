\subsection{Horizontal origins of Taumatins in nematodes and insects}
Taumatins are a group of defensive proteins produced by plants in response to
fungal infections [link]. Recently the presence of these proteins was reported
in the desert locust \textit{Schistocerca gregaria}, one related species, and
in the nematode \textit{Caenorhabditis} but not in \textit{Drosophila} and
also some insects \cite{Brandazza2004}. Our analysis suggests that the ancestors of
\textit{Caenorhabditis elegans} may acquired Thaumatin proteins via HGT.
According to InterPRO \cite{Finn2017} there are currently 4853 known Thaumatin
proteins (IPR001938). 3179 of them are found in land plants, 1120 in Fungi and
only 118 in Metazoa. The latter 118 are distributed among nematodes (59
proteins), panarthropods (57 proteins) and mollusks (2 proteins). This
distribution of Thaumatin proteins among the tree of life favors the idea that
HGT was involved in their evolution.
