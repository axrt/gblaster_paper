\subsection{Horizontal origins of Taumatins in nematodes and insects}
Taumatins are a group of defensive proteins produced by plants in response to
fungal infections, some of which are characterized as “sweet” by humans.
Recently the presence of these proteins was reported in the desert locust
Schistocerca gregaria, a related species Locusta migratoria and in the nematode
Caenorhabditis but not in Drosophila and other insects \cite{Brandazza2004}. Our analysis
suggests that Tribolium castaneum and Caenorhabditis elegans have Thaumatin
proteins, which were previously acquired via HGT.  According to InterPRO
\cite{Finn2017}
there are currently 4853 known Thaumatin proteins (IPR001938). 3179 of them are
found in land plants, 1120 in Fungi and only 118 in Metazoa. These 118 are
distributed between nematodes (59 proteins), panarthropods (57 proteins) and
mollusks (2 proteins). This distribution of Thaumatin proteins among the tree
of life favors the idea that HGT was involved in their evolution.
