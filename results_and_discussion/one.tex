\subsection{Outsider detection}
We performed 61*60/2 = 1830 in silico metagenomic experiments involving a total
of 296639 predicted protein-coding sequences, each of which had a hit in at
least 3 other genomes. A total of 67\% of the obtained regression models were
successfully validated and thus allowed the distinction between the sequences
of two genomes. Let us use the Drosophila simulans/Wolbachia in silico
metagenomic experiment as an example of what was done. Figure 1 shows a
principal component analysis (PCA) visualization of sequence clustering based
on the normalized values of similarity scores for the Drosophila simulans and
Wolbachia metagenomic experiment. A clear separation of sequences can be
observed even without the use of regression analysis by PCA alone; however, we
used PCA analysis only for illustrative purposes. The majority of sequences
(288012, 97.1\%) from each analyzed genome were classified as core sequences in
all metagenomic experiment with validated regression models. 4279 (1.4\%) of
the sequences were classified as “outsider once”. The remaining 4348 (1.5\%)
sequences were classified as “outsider multiple” and are the most likely
candidates for HGT. The list of protein-coding open reading frames classified
as “outsider once”, or “outsider multiple” along with their FASTA sequences is
available for download: link.The fraction of “outsider once” and “outsider
multiple” genes varies from genome to genome, as shown on Figure 4.An important
limitation is that our approach does not distinguish HGT from genomic
contaminations. Thus the observed differences in Core/Outsider ratio between
different genomes, does not necessarily indicate that some genomes are more
prone to the acquisition of sequences due to HGT. We used a PFAM analysis to
check for any PFAM domains are significantly overrepresented in “outsider
multiple” sets of sequences. A list of top-ranked findings (ranked by
significance) is presented in Table 1.We would like to highlight some of the
most interesting findings.
