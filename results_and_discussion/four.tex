\subsection{HGT genes detected in Archaeon Loki}
\textit{Archaeon Loki} was reported as a complex archaea that bridges the gap
between prokaryotes and eukaryotes \cite{Spang2015}. Our initial analysis
revealed sequences of one domains that are overrepresented among candidates for
HGT in \textit{Archaeon Loki}: MaoC\_dehydratas, Ras, Roc, Arf, LRR\_8. For
example, the Maoc\_dehydratase domain is present in 7 PPS of \textit{Archaeon
Loki}. All of them were classified as "outsider multiple" by our model.
Meanwhile only 64 PPS of Archaeon Loki have this classification. 79 PPS are
classified as "outsider once" and 1409 as "always core". All 7 Maoc\_dehydratas
containing proteins of \textit{Archaeon Loki} are not similar to any proteins
found in archaea, but share some degree of similarity to proteins found in
d-proteobacteria according to a simple BlastP search (e-values up to E-15).
Ras proteins are involved in cellular differentiation, proliferation and
survival and are present in the vast majority of metazoa, but are rarely found
in prokaryotes (IPR001806). In a similar way Roc and Arf domains are universal
for eukaryotes, not prokaryotes. For example, see the distribution among
different groups of organisms for Arf (PFAM: PF00025).
