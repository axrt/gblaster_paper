\subsection{Horizontal origin of Ankyrins in Wolbachia endosymbiotic bacteria}
\label{horizontal_origin}
Jernigan and Bordenstein \cite{Jernigan2014} showed that the lifestyle of
bacteria, rather than phylogenetic history, is a predictor of ANK repeat
abundance. They also showed that phylogenetically unrelated organisms that
forge facultative and obligate symbioses with eukaryotes show enrichment for
ANK repeats in comparison to free-living bacteria. This observation was
especially strong for obligate intracellular bacteria. Ankyrin domains are very
common in eukaryotes, but rare in bacteria, with the exception of parasites and
symbionts. In a paper by Siozios et al. \cite{Siozios2013} it is concluded that
ankyrin genes are likely to be horizontally transferred between strains with
the aid of bacteriophages. Al-Khodor et al. \cite{Al-Khodor2010} also suggests
that prokaryotic genes encoding ANK-containing proteins have been acquired from
eukaryotes by horizontal gene transfer. Coincidentally we independently
reproduced these findings: both species of endosymbiotic Wolbachia included in
our study have genes encoding ANK-containing proteins acquired via HGT by their
ancestors according to our models.
